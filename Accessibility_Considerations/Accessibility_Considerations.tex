\documentclass{article}
\usepackage{graphicx}
\usepackage{hyperref}

\title{Accessibility Considerations}
\begin{document}
\maketitle
\paragraph{
The PreTeXt framework is responsible for many accessibility features. Read more about the accessibility features on the \href{https://pretextbook.org/doc/guide/html/topic-accessibility.html}{PreTeXt documentation}. These accessibility considerations come from the Web Content Accessibility Guidelines (WCAG) international standard.
}

\section{Colors}
\paragraph{
HTML elements and attributes may be interpreted by a screen reader and pass validation checks. Distinguishing between elements does not rely exclusively on color.
}

\section{Mathematics}
\paragraph{
The MathJax JavaScript library allows screen readers to interpret the HTML.
mathjax.org
}

\section{3D Images}
\paragraph{
The Asymptote language describes 2D and 3D images.
}

\section{Navigating Content}
\paragraph{
Pressing the Tab key moves a reader across locations in the web document.
}

\section{Links}
\paragraph{
Hyperlinks are consistent with WCAG recommendations.
}

\section{Image Description}
\paragraph{
Screen readers pick up a short description, which migrates to the HTML @alt attribute and a description element.
}

\section{Cross-References}
\paragraph{
Larger target for clickable items.
}

\section{Link Text}
\paragraph{
Alternative content for the clickable portion of link text.
}

\section{Conversion to Brail}
\paragraph{
For more information see the \href{https://pretextbook.org/doc/guide/html/publisher-braille.html}{documentation}.
}

\section{Conversion to PDF and EPUB}
\paragraph{
PDF and EPUB conversion available for off-line use.
}
\end{document}
