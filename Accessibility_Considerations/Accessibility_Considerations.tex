\documentclass{article}
\usepackage{graphicx}
\usepackage{hyperref}

\title{Accessibility Considerations}
\author{Zunaid Ahmed, Allaoua Boughrira, Hellen Colman, Samuel Lubliner}
\begin{document}
\maketitle
Much of the technical work for accessibility is accomplished by PreTeXt, but we also took the responsibility as authors of testing as part of our workflow. PreTeXt is ideally positioned to help us easily create highly-accessible documents since they are interested in helping authors produce documents with open licenses, and concentrated on employing open standards for the HTML output created.

The PreTeXt framework is responsible for many accessibility features. Read more about the accessibility features on the \href{https://pretextbook.org/doc/guide/html/topic-accessibility.html}{PreTeXt documentation}. These accessibility considerations come from the Web Content Accessibility Guidelines (WCAG) international standard.


\section{Colors}

HTML elements and attributes may be interpreted by a screen reader and pass validation checks. Distinguishing between elements does not rely exclusively on color.


\section{Mathematics}

The MathJax JavaScript library allows screen readers to interpret the HTML.
mathjax.org


\section{3D Images}

The Asymptote language describes 2D and 3D images.


\section{Navigating Content}

Pressing the Tab key moves a reader across locations in the web document.


\section{Links}

Hyperlinks are consistent with WCAG recommendations.


\section{Image Description}

Screen readers pick up a short description, which migrates to the HTML @alt attribute and a description element.


\section{Cross-References}

Larger target for clickable items.


\section{Link Text}

Alternative content for the clickable portion of link text.


\section{Conversion to Braille}

For more information see the \href{https://pretextbook.org/doc/guide/html/publisher-braille.html}{documentation}.


\section{Conversion to PDF and EPUB}

PDF and EPUB conversion available for off-line use.

\end{document}
